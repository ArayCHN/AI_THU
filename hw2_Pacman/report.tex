\documentclass{article}
\usepackage{graphicx} % include figures
\usepackage{xeCJK} % Chinese language support
\usepackage{bm}
\usepackage{amsmath,amsthm,amssymb,amsfonts}
\usepackage{cite}
\usepackage[colorlinks,linkcolor=red,anchorcolor=blue,citecolor=green,CJKbookmarks=true]{hyperref}
\usepackage{indentfirst} % indent before a paragraph, Chinese-style
\usepackage{amsmath}
\usepackage[margin=3.5cm]{geometry}
\usepackage{titlesec}
\usepackage{amsmath}
\usepackage{amssymb}
% \linespread{1.6}
\geometry{left=3.2cm,right=3.2cm,top=3.2cm,bottom=3.2cm}
\usepackage{booktabs}
\usepackage{multirow}
\usepackage{array} %为了能给表格指定宽度
\usepackage{listings}
\usepackage{xcolor}
\usepackage{ulem}
\usepackage{enumitem}
\usepackage{tikz}
\usepackage{lipsum}

%\usepackage{mdframed} % 为了在代码周围给出边框,换页时边框保持完整
\setenumerate[1]{itemsep=0pt,partopsep=0pt,parsep=\parskip,topsep=5pt}
\setitemize[1]{itemsep=0pt,partopsep=0pt,parsep=\parskip,topsep=5pt}
\setdescription{itemsep=0pt,partopsep=0pt,parsep=\parskip,topsep=5pt}
%定理
\makeatletter
\thm@headfont{\sc}
\makeatother
\newtheorem{theorem}{Theorem}
%%%%%%%%%%%%%%%%%%%%%%%%%%%%%%
\title{THU AI Assignment II: Berkeley Pacman Project 2 \\ [2ex] \begin{large} Multi-Agent Search \end{large} }
\author{\large Rui\hspace{0.2cm}Wang\footnote{wangrui15@mails.tsinghua.edu.cn} \\ 2015010445}
\date{}
\begin{document}
\maketitle

\section{Q1: Reflex Agent}
  \subsection{Determine the Evaluation Function}
  The class \textit{ReflexAgent} mainly applies an evaluation function to estimate the best strategy every step. Basically, we consider the component of a game state. We need to take into consideration several factors:
  \begin{itemize}
    \item The possibility of being eaten by ghosts, denoted by the distance to ghosts.
    \item The chance of eating a dot, denoted by the distance to the nearest dot.
  \end{itemize}\par
  One thing worth noting is that in a state, pacman will have already eaten the dot if it is at a dot. Therefore, when we want to evaluate a new state lead to by a certain action, we need to calculate the distance between \emph{NEW} pacman position and the closest dot in the \emph{CURRENT} state.\par
  The evaluation function I chose is as follows. Let $d_i$ be the distance between pacman and ghost number $i$, $d_{food}$ be the shortest manhattan distance between dot and pacman, $P$ be the penalty of taking a specific step while $B$ be the bonus of taking it. We define:
  \[ P = \sum_{i} \frac{k_1}{(d_i - k_2)^{k_3}} \] where $k_j$ are constants(parameters) to be determined through test and practice. For reference, in my version of implementation, $k_1 = 3$, $k_2 = 0.5$, $k_3 = 2$.
  \[ B = \frac{m_1}{d_{food} + m_2}\] Also in my code, $m_1 = 2$, $m_2 = 0.2$.\par
  \subsection{Test Result}
  The test result of these parameters proves promising, with an average score of $1236.5$. This result shows the validity of such evaluation.
  
\section{Q2: Minimax}
  \subsection{Summary}
  Since a stack-simulated recursive search would prove too complicated for this algorithm, a recursive function that calls itself is defined in the implementation of this Minimax algorithm.
  \subsection{Special Discussion}
  \textbf{\emph{Notice:}} Huge pitfall: a very important thing to notice is that the assigned searching depth for pacman might result in no choice for it, i.e. the function \textit{getLegalActions} may end up returning no action at all. In this case, we need to break the exploration in this branch at once or we will end up returning the \textit{maxScore} as -99999 or \textit{minScore} as 99999, as they are initialized.\par
  This bug in particular cost me more than 3 hours to find out. Actually the \textit{autograder.py} will return a searching tree with its depth and assigned depth not matching. This is, however, legal case since depth is '\textbf{ARBITRARY}'.
  \subsection{Test Results} 
  Run the following command, and we will find:
\begin{verbatim}
    python pacman.py -p MinimaxAgent -a depth=3 -l smallClassic
\end{verbatim}\par
  In the case of trapped classic maze, pacman will bump into the closest ghost since it is calculating its grade several steps later. This is not always true especially in a case where ghost behavior is unpredictable(and often enough suboptimal).
  In the case of a small classic maze, pacman takes around 0.5 seconds to make a move(on a macOS system, with 8GB memory).

\section{Q3: $\alpha-\beta$ Pruning}
  \subsection{Summary}
  The most important part of $\alpha-\beta$ Pruning is the algorithm itself. It is not as trivial as it seems at first sight. In a typical search tree, in order to keep trace of nodes expanded earlier than the direct parent, we need to pass two parameters, $a$ and $b$ to a node. The specific method is described in detail in lecture note.
  \subsection{Test Result and Discussion}
  Running $\alpha-\beta$ Pruning on a classic small maze with the following command:
\begin{verbatim}
    python pacman.py -p AlphaBetaAgent -a depth=3 -l smallClassic
\end{verbatim}\par
    Pacman will be running a bit faster than with a traditional minimax agent. However, improvement is not satisfactory. This implies that $\alpha-\beta$ Pruning cannot bring substantial improvement to a searching algorithm.
    
\section{Q4: Expectimax}
\textit{ExpectimaxAgent} simply calculates all possible moves of a ghost. There's not much to be altered in the code of \textit{MinimaxAgent}. Test result shows that pacman can survive half the time in a \textit{trappedClassic} maze. It also wins the \textbf{minimaxClassic} maze which \textit{minimaxAgent} fails to win.

\section{Q5: Better Evaluation Function}
This evaluation function here is called by the \textit{autograder} to make evaluation for different states. There is, however, some difference between a state-estimation function and an action-estimation function. For example, when estimating whether a state is good or not, we should consider the total number of food remaining instead of the distance to the closest food, because the state even later should not affect pacman's evaluation of the state for the time being, otherwise pacman would often keep static even if a food is close to it-since eating the food will only result in a worse state estimation.\par
My implementation of state estimation consists of the following factors:
\begin{itemize}
  \item If the state is win/lose, will go to/avoid it.
  \item The (unscared) ghosts should be kept as far away as possible.
  \item The food remaining should be as little as possible.
\end{itemize}
Denote $p$ as penalty and $b$ as bonus, and then define
$$ p = - \sum_{i} \frac{5}{(d_i - 0.5)^2} $$ where $d_i$ id the distance to the $i^{th}$ ghost.
$$ b = \frac{1}{(d_i - 0.5)^2} - 5n - 100m $$ where $n$ is the number of food remaining and $m$ is the number of unscared ghosts.
Finally return the following result as the evaluation of this state:
$$ p + b + currentGameState.getScore() $$
The last item $currentGameState.getScore()$ is extremely tricky. Adding it will result in an eligible agent, while lacking it will result in strange behavior that is against common sense. However the root of this problem is still unknown since there is no clue of how this function is called. \emph{Thus students should be reminded of this detail which otherwise would be impossible to find out.}
\end{document}